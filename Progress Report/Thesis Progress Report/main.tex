\documentclass[a4paper,11pt]{article}
\usepackage[margin = 2.54cm]{geometry}   
\usepackage[utf8]{inputenc}
\usepackage{graphicx}
\usepackage[export]{adjustbox}
\usepackage{hyperref}
%\usepackage{preamble}
\usepackage{float}
\usepackage{amsfonts}
\usepackage{amsmath}
\usepackage{siunitx}
\usepackage{subcaption}
\usepackage{hyperref}
\usepackage{titlesec}
\usepackage{setspace}
\usepackage{verbatim}
\usepackage{mathabx}
\usepackage{pdfpages}
\usepackage{geometry}
\usepackage{booktabs}
\usepackage{multicol}
\usepackage{multirow}
\usepackage{colortbl}
\usepackage{tabulary}
\usepackage{tabularx}
\usepackage{array}
\usepackage{longtable}
\usepackage{etoolbox}
\usepackage{ragged2e}
\usepackage{wrapfig}
\usepackage{pythonhighlight}
\usepackage[sorting=none]{biblatex}
\addbibresource{references.bib}
\usepackage{hyperref}

\title{}
%\linespread{1}

\parindent = 0pt % No paragraph indent.

% Renames the title of the table of contents from "Contents" to "Table of Contents"
\renewcommand*\contentsname{Table of Contents} 

% URL link colours should be blue.
\hypersetup{
  colorlinks = true,
  linkcolor = black,
  urlcolor = blue,
  citecolor = black
}


% ----------------------------------------------------------------------------
%                                 Document
% ----------------------------------------------------------------------------
\begin{document}

% ----------------------------------------------------------------------------
%                                 Title Page
% ----------------------------------------------------------------------------
\begin{titlepage}
% Title, Table of Content, Acronyms, Nomenclature, Executive Summary. The title should be informative and concisely explain the presented work. Often the title carries the keywords of the presented work. The table of content should list all sections and subsections. The acronyms section should list in alphabetical order all the acronyms used in the report. All symbols, their meaning and units (if applicable) should be listed in the nomenclature section. The executive summary (no more than 1 page) should summarise the motivation/problem statement, its importance, and how the presented work (successfully) addressed the problem.

% USYD logo
\newcommand{\HRule}{\rule{\linewidth}{0.5mm}} 
\begin{flushright}
\includegraphics[width=0.3\textwidth]{Images/usydNew.png}\\[2cm]
\end{flushright}

\center 

% Titles 
\textsc{\LARGE The University of Sydney}\\[0.75cm]
\textsc{\Large School of Aerospace, Mechanical, Mechatronic, and Electrical Engineering}\\[1cm]

\textsc{\Large}\\[0.25cm] 

\HRule \\[0.4cm]
{ \huge \bfseries Aeroelastic Flutter Suppression}\\[0.4cm] 
{\Large Honours Thesis Progress Report}\\[0.4cm]
\HRule \\[1.25cm]

% AUTHORS
\begin{minipage}{0.4\textwidth}
\begin{flushleft}
\emph{Author:}\\
\textsc{Thomas Lando}\\{SID: 490388538}\\[0.2cm]

\end{flushleft}
\end{minipage}
~
\begin{minipage}{0.4\textwidth}
\begin{flushright}\large\emph{}\\
\end{flushright}
\end{minipage}\\[0.8cm]
\vfill
{\large \today}\\[0.8cm]


\vfill
\end{titlepage}
% ----------------------------------------------------------------------------
\section{Introduction}
% Introduce the thesis
 
\begin{itemize}
    \item{Define aeroelastic flutter}
    \item{introduce history behind the mathematics and general history of analysis}
    \item{talk about reasons for trying to suppress flutter and possible uses for an active supression system.}
\end{itemize}

% ------------------------------------------------------------------
\section{Literature Review}
\subsection{Aerodynamics}

\subsubsection{Principles of Aeroelasticity (1962)}
Written by Raymond L. Bisplinghoff and Holt Ashely from MIT, this book helped to source the aeroelastic equations of motion and also provided a reasonably detailed process on how to solve for the flutter velocity in one, two, and three dimensions.  This piece of literature also provided many references to find more detail and further explanations into how the general process of solving for the flutter velocity is done.  One notable example was a reference to NACA Report  496 which outlined a very detailed process on solving the equations of motion.  After reading this piece of literature I was able to begin forming the process for the theoretical simulation and work out where the remaining gaps were.  For example, I required more information to solve for $C(k)$. 

\subsubsection{Introduction to the Study of Aircraft Vibration and Flutter (1951)}
A book by Scanlan, this peice of literature like the one mentioned above 

\subsubsection{NACA Report 496: General Theory of Aerodynamic Instability and the Mechanism of Flutter (1949)}

\subsection{Flutter Suppression Method}

\subsubsection{Active Control Method On Flutter Suppression Of a
High-Aspect-Ratio Two-Dimensional Airfoil with a Control Surface (2014)}

\subsubsection{Active Flutter Suppression in Aircraft Wings (1992)}

\subsubsection{Aeroelastic Control of Flutter using Trailing Edge Control Surfaces powered by Piezoelectric Actuators (2003)}
\subsection{Other}


% ------------------------------------------------------------------
\section{Summary of Completed Work}
\subsection{2D 3DOF Theoretical Model}
\begin{itemize}
    \item 
\end{itemize}
\subsection{2D 3DOF Nastran Model}

\subsection{3D 6DOF Nastran Model}
% ------------------------------------------------------------------
\newpage
\section{References}
\begin{enumerate}
    \item{R.L. Bisplinghoff, "Flutter", "Principles of Aeroelasticity", 1st Edition, New York, United States of America, John Wiley and Sons INC., Chpt 9,pp. 527-631 }
\end{enumerate}
\newpage
\appendix
\section{2D 3DOF Theoretical State Space Derivation}

\section{2D Nastran Model Mesh Process}



\end{document}
